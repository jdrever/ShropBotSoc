\documentclass[a4paper,12pt,landscape]{article}
\usepackage{fullpage} % for 1.5 cm margins
\renewcommand{\familydefault}{\sfdefault} % so it doesn't look like LaTeX
\usepackage{helvet}
\usepackage[british]{isodate}
\usepackage{abstract}
\renewcommand{\abstractname}{Overview}
\newcommand{\tickbox}{\framebox(12,12){} }
\newcommand{\fillbox}{\begin{tabular}{|l|}
  \hspace{60pt} \\
  \hline
\end{tabular} }
\usepackage{enumitem, amssymb} % for checklists
\newlist{todolist}{itemize}{2}
\setlist[todolist]{label=$\tickbox$, parsep=1em}
\raggedright
\raggedbottom
\usepackage{multicol}
\usepackage{graphicx}
\usepackage{parskip}

\newcommand{\documenttitle}{Shropshire Botanical Society Online Flora}
\newcommand{\documentauthor}{Joe J Collins}

\title{Shropshire Botanical Society Online Flora\\
Web application Specification}
\author{\documentauthor}
\date{\today}

\usepackage[pdftex,
  pdftitle={\documenttitle}, 
  pdfauthor={\documentauthor},
  pdfsubject={\documenttitle}]{hyperref}

\begin{document}
\maketitle


\begin{abstract}
  \begin{center}
    \begin{minipage}{0.5\textwidth}
      \strut\\
      The Shropshire Botanical Society is seeking
      to renew it's Online Flora web application.
      This specification out lines the hoped for functionality
      together with the technical
      and
      development constraints of the work.
    \end{minipage}
  \end{center}
\end{abstract}

\clearpage
\begin{multicols*}{2}
  \tableofcontents
\end{multicols*}
\clearpage


\begin{multicols*}{2}
  \section{Background}
  The Shropshire Botanical Society
  has been dedicated to promoting the enjoyment,
  understanding and conservation of the flora of Shropshire
  since the 18\textsuperscript{th} century.
  One of the principle activities of the Society is to collect and maintain records 
  of plant sightings within the historical boundaries of the county of Shropshire.
  Since 2003 the Society has made these records freely available online via a bespoke web application
  or Online Flora.
  This original Online Flora was written using
  PHP and \href{https://codeigniter.com/}{CodeIgniter Web Framework}
  backed by MySql database.
  The web application is still available at 
  \href{https://captain-blue.azurewebsites.net/}{captain-blue.azurewebsites.net}
  but unfortunately the data is now many years out of date.

  Maintaining and updating the database has proved to be challenging.
  Additionally the application was conceived prior to the introduction of the iPhone
  and it not suited to mobile use.
  Hence the Society seeks to renew the web application,
  to provide a more modern mobile interface
  and to use up to date data stored
  by the \href{https://nbnatlas.org/}{National Biodiversity Network Atlas}.
  Currently all the Society's records are submitted to the 
  National Biodiversity Network Atlas
  and since 2017 the Society's records have been available via a web service at
  the \href{https://api.nbnatlas.org/}{NBN Web service API}.
  Using the NBN Web service API provides reliable data source
  and
  a supported service for maintaining and updating the Society's records.

  \vfill\strut\columnbreak

  \section{Objective}
  To replicate the functionality of the original Online Flora
  in a responsive mobile design
  using data sourced from the NBN Web service API.

  \section{Usage and Users}
  The Online Flora is used for searching the Society's records
  but not for entering new records.
  Maintaining and updating the data is conducted via a separate manual process.
  Searches of the database are conducted for three different geographical scenarios.

  \begin{description}
      \item[Search Shropshire]
        searching all the records of based on the name of the plant.
        Allowing the user to drill down to a single sighting record
        or
        showing a map of grid squares with records for a named plant.
      \item[Search by Site]
        searching for a named site,
        then listing the names of plants for that named site.
        Again allowing the user to drill down to a single sighting record.
      \item[Search by Monad or Grid Square]
        Selecting a 1 km grid square within the county of Shropshire,
        then listing the names of plants for that named site.
        Again allowing the user to drill down to a single sighting record.
  \end{description}

  Users of the Online Flora are typically
  members of the Society
  and
  as such are often very experienced botanists
  and will favour identifying
  plant species via the scientific name.
  A typical scenario would be a member of the Society
  intending to visit a location
  would search for a list of plants that have previously
  been sighted at that location.
  A similar search might also be
  conducted at the location of interest using a mobile phone.
  Up to this time the Society has not offered an interface that is suitable for mobile phone use.
  As a result there is no information about the types and sizes of the devices that might be used.
  However the Society does maintain a blog/website at 
  \href{https://www.shropshirebotany.org.uk/}{www.shropshirebotany.org.uk}.
  The majority of visitors to the site
  use Google Chrome on a Windows platform.
\end{multicols*}

\begin{multicols*}{2}[%
  \section{Search County by Plant Name}%
  \subsection{Mobile}%
]
\includegraphics[width=0.85\textwidth]{./wireframes/county-mobile.png}%
\clearpage

  Search within the entire collection for the county.

  \begin{description}
      \item[Scientific Name] selected by default.
      \item[Common Name] where common name is selected only those sightings with a common name will be show.
      \item[Axiophytes] 
  \end{description}
  
\end{multicols*}

\subsection{Desktop}
\includegraphics[width=0.6\textwidth]{./wireframes/county-desktop.png}

   \begin{verbatim}
    https://records-ws.nbnatlas.org/explore/group/Birds?fq=data_resource_uid:dr782+AND+taxon_name:B*&pageSize=12
   \end{verbatim} 

\clearpage
\section{Search Grid Square by Plant Name}
\subsection{Mobile}

    \includegraphics[width=0.9\textwidth]{./wireframes/monad-mobile.png}


\section{Technical Constraints}

Simplest possible with lowest barrier to entry.
Success with PHP and CodeIgniter.
Simple and convenient caching.
Possible to source PHP developers from within the ranks of members.
Possibility of free hosting.

\begin{description}
    \item[PHP 7.3] for deployment to Google App Engine.
    \item[CodeIgniter 4.0.4] happ with 
        No Caching to reveal performance, convenient caching later.
    \item[Twitter Bootstrap 5] for responsive layout.
    \item[Leaflet 1.6.0] Mapping https://github.com/DuncanRowland/NBNMapOverlayExamples
    \item[No NBN API Calls from the Client] because we might want to use caching.
    \item[Commits to Github] branching but no where else.
        at \href{https://github.com/joejcollins/captain-magenta.git}{Github}
      The Society will
    \item[Style Sheet] from the blogspot Website
    \item[JavaScript] plain, no jquery.
    \item[No database] but you could have static data files.  
\end{description}

\end{document}


